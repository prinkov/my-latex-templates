% 	Name		:: 	sthlm Beamer Theme  HEAVILY based on the hsrmbeamer theme (Benjamin Weiss)
%	Author		:: 	Mark Hendry Olson (mark@hendryolson.com)
%	Created		::	2013-07-31
%	Updated		::	June 18, 2015 at 08:45
%	Version		:: 	1.0.2
%	Email		:: 	mark@hendryolson.com
%	Website		:: 	http://hendryolson.com
%
% 	License		:: 	This file may be distributed and/or modified under the
%                  	GNU Public License.
%
%	Description	::	This presentation is a demonstration of the sthlm beamer
%					theme, which is HEAVILY based on the HSRM beamer theme created by Benjamin Weiss
%					(benjamin.weiss@student.hs-rm.de), which can be found on GitHub
%					<https://github.com/hsrmbeamertheme/hsrmbeamertheme>.


%-=-=-=-=-=-=-=-=-=-=-=-=-=-=-=-=-=-=-=-=-=-=-=-=
%
%        LOADING DOCUMENT
%
%-=-=-=-=-=-=-=-=-=-=-=-=-=-=-=-=-=-=-=-=-=-=-=-=

\documentclass[newPxFont,sthlmFooter]{beamer}
\usetheme{sthlm}
%\usecolortheme{sthlmv42}

%-=-=-=-=-=-=-=-=-=-=-=-=-=-=-=-=-=-=-=-=-=-=-=-=
%        LOADING PACKAGES
%-=-=-=-=-=-=-=-=-=-=-=-=-=-=-=-=-=-=-=-=-=-=-=-=
\usepackage[utf8]{inputenc}
\usepackage[russian]{babel}
\usepackage{chronology}
\usepackage{booktabs}
\usepackage[scale=2]{ccicons}
\usepackage{pgfplots}
\usepgfplotslibrary{dateplot}
\usepackage{xspace}
\usepackage{times}
\newcommand\tab[1][1cm]{\hspace*{#1}}
\usepackage{ragged2e}
\usepackage{amssymb}
\usepackage{amsmath}
\usepackage{graphicx}
\usepackage{mathtools}
\usepackage{diagbox}
\usepackage{amsmath} % для системы
\usepackage{tikz}
\usepackage{tikz-cd}
\usepackage{tikz-qtree}
\usepackage{tabularx}
\usepackage{qtree} 
\usepackage{bbm}
\usepackage{chronology}
%\usepackage{bbold}
\DeclareMathAlphabet{\mathbbold}{U}{bbold}{m}{n}
\usepackage{setspace}
\usepackage{scalerel}
\renewcommand{\event}[3][e]{%
  \pgfmathsetlength\xstop{(#2-\theyearstart)*\unit}%
  \ifx #1e%
    \draw[fill=black,draw=none,opacity=0.5]%
      (\xstop, 0) circle (.2\unit)%
      node[opacity=1,rotate=45,right=.2\unit] {#3};%
  \else%
    \pgfmathsetlength\xstart{(#1-\theyearstart)*\unit}%
    \draw[fill=black,draw=none,opacity=0.5,rounded corners=.1\unit]%
      (\xstart,-.1\unit) rectangle%
      node[opacity=1,rotate=45,right=.2\unit] {#3} (\xstop,.1\unit);%
  \fi}%

%-=-=-=-=-=-=-=-=-=-=-=-=-=-=-=-=-=-=-=-=-=-=-=-=
%        BEAMER OPTIONS
%-=-=-=-=-=-=-=-=-=-=-=-=-=-=-=-=-=-=-=-=-=-=-=-=

%\setbeameroption{show notes}

%-=-=-=-=-=-=-=-=-=-=-=-=-=-=-=-=-=-=-=-=-=-=-=-=
%
%	PRESENTATION INFORMATION
%
%-=-=-=-=-=-=-=-=-=-=-=-=-=-=-=-=-=-=-=-=-=-=-=-=


\newcommand{\A}{\mathcal{A}}
\newcommand{\N}{\mathbb{N}}
\newcommand{\Z}{\mathbb{Z}}
\newcommand{\Zp}{\mathbb{Z}_p}
\newcommand{\ZpZ}{\mathbb{Z}/p\mathbb{Z}}
\newcommand{\ZnZ}{\mathbb{Z}/n\mathbb{Z}}
\newcommand{\Fpn}{\mathbb{F}_{p^n}}
\newcommand{\Qh}{\mathcal{Q}_h}
\newcommand{\X}{\mathbb{X}}
\renewcommand{\C}{\mathbb{C}}
\newcommand{\Cp}{\mathbb{C}_p}
\newcommand{\R}{\mathbb{R}}
\newcommand{\Rmin}{\mathbb{R}_\text{min}}
\newcommand{\Rmax}{\mathbb{R}_\text{max}}
\newcommand{\Rminmax}{\mathbb{R}_\text{min,max}}
\newcommand{\Rmaxmin}{\mathbb{R}_\text{max,min}}
\newcommand{\Q}{\mathbb{Q}}
\newcommand{\Qp}{\mathbb{Q}_p}
\newcommand{\Qpmin}{\mathbb{Q}_{p,\text{min}}}
\newcommand{\Qpmax}{\mathbb{Q}_{p,\text{max}}}
\newcommand{\Qpminmax}{\mathbb{Q}_{p,\text{min,max}}}
\newcommand{\Qpmaxmin}{\mathbb{Q}_{p,\text{max,min}}}
\newcommand{\Tp}{\mathbb{T}_p}
\newcommand{\z}{\mathbbold{0}}
\renewcommand{\u}{\mathbbold{1}}
\newcommand{\comma}{\scaleobj{2}{,}}

\newcolumntype{H}{>{\centering\arraybackslash}p{2em}}
\newcolumntype{Q}{>{\centering\arraybackslash}p{5.9em}}
\newcolumntype{C}{>{\centering\arraybackslash}p{8em}}
\newcolumntype{D}{>{\centering\arraybackslash}p{11em}}
\newcolumntype{L}{>{\centering\arraybackslash}p{13em}}

\newcommand*{\hindent}{\hspace*{-0.1cm}}
\newcommand*{\vindent}{\vspace*{-0.5cm}}
\newcommand*{\vvindent}{\vspace*{-0.7cm}}


\DeclarePairedDelimiterX{\norm}[1]{\lVert}{\rVert}{#1}
\apptocmd{\frame}{}{\justifying}{}

% если нужны блоки
% \setbeamertemplate{blocks}[rounded][shadow=true]
% \setbeamercolor*{title}{use=structure,fg=white,bg=structure.fg,}
% \setbeamertemplate{title page}[default][colsep=-4bp,rounded=true,shadow=true]

\tikzset{
   commutative diagrams/.cd,
   arrow style=tikz,
   diagrams={>=latex}}
\tikzcdset{  crossing over/.style={
    /tikz/preaction={
      /tikz/draw,
      /tikz/color=\pgfkeysvalueof{/tikz/commutative diagrams/background color},
      /tikz/arrows=-,
      /tikz/line width=\pgfkeysvalueof{/tikz/commutative
      diagrams/crossing over clearance}}}}


\title{$p$-адическая интерполяция дзета-функции Римана}
\subtitle{}
%\date{\small{\jobname}}
%\date{\today}
\author{\texttt{Приньков А. С.}}
\institute{Липецк 2016}

% \hypersetup{
% pdfauthor = {Mark H. Olson: mark@hendryolson.com},
% pdfsubject = {},
% pdfkeywords = {},
% pdfmoddate= {D:\pdfdate},
% pdfcreator = {}
% }

\begin{document}

%-=-=-=-=-=-=-=-=-=-=-=-=-=-=-=-=-=-=-=-=-=-=-=-=
%
%	TITLE PAGE
%
%-=-=-=-=-=-=-=-=-=-=-=-=-=-=-=-=-=-=-=-=-=-=-=-=

\maketitle

% \begin{frame}[plain]
% 	\titlepage
% \end{frame}

%-=-=-=-=-=-=-=-=-=-=-=-=-=-=-=-=-=-=-=-=-=-=-=-=
%
%	TABLE OF CONTENTS: OVERVIEW
%
%-=-=-=-=-=-=-=-=-=-=-=-=-=-=-=-=-=-=-=-=-=-=-=-=

\section*{sthlm Theme Information}

%-=-=-=-=-=-=-=-=-=-=-=-=-=-=-=-=-=-=-=-=-=-=-=-=
%	FRAME:
%-=-=-=-=-=-=-=-=-=-=-=-=-=-=-=-=-=-=-=-=-=-=-=-=

\begin{frame}[c]{Цель работы}
Целью работы является исследование вопроса сходимости дзета-функции Римана и её практического применения.
Для достижения поставленной цели сформулированы следующие задачи:
\begin{itemize}
\item Изучить литературу по $p$-адическому анализу
\item Применить на практики методологию $p$-адической интерполяции
\item Провести практические вычисления
\end{itemize} 

\end{frame}

%-=-=-=-=-=-=-=-=-=-=-=-=-=-=-=-=-=-=-=-=-=-=-=-=
%	FRAME:
%-=-=-=-=-=-=-=-=-=-=-=-=-=-=-=-=-=-=-=-=-=-=-=-=

\begin{frame}[c]{Основные определения}
\begin{alertblock}{Определение нормы}
Нормой на поле F называется отображение, обозначаемое через $\| \|$, поля F в множество неотрицательных вещественных чисел, такое что:
\begin{enumerate}
\item $\|x\| = 0 \Leftrightarrow x = 0$
\item $\|x \odot y\| = \|x\| \odot \|y\|$
\item $\|x + y \| \leq \|x\| + \|y\|$
\end{enumerate}
\end{alertblock}


\end{frame}

%-=-=-=-=-=-=-=-=-=-=-=-=-=-=-=-=-=-=-=-=-=-=-=-=
%	FRAME:
%-=-=-=-=-=-=-=-=-=-=-=-=-=-=-=-=-=-=-=-=-=-=-=-=

\begin{frame}{Основные определения}
\begin{alertblock}{Определение метрики}
Функция $d$, определенная на множестве пар $(x, y) \in F$ и принимающая значения из $\mathbb{R}_+$, называется метрикой в $F$, если она обладает следующими свойствами:
\begin{enumerate}
\item $ d(x, y) = 0 \Leftrightarrow x = y$
\item $d(x, y) = d(y, x)$
\item $d(x, y) \leq d(x, z) + d(z, y)$
\end{enumerate}
\vspace{0.5em}
Метрика $d$ инуцирована нормой, если $d(x,y) = \|x-y\|$ 
\end{alertblock}

\end{frame}

\section*{Overview}

\begin{frame}[fragile]{$p$-адическое расширение}\vvindent
Определение нормы $\mathbb{Q}_p$
\[
|x|_p=
\left\{
\begin{aligned}	
&0,& x = 0 \\
&\frac{1}{p^{ord_pa}},& x \ne 0 \\
\end{aligned}
\right.
\]

\begin{block}{Расширение алгебраических структур (Гамильтон, 1832)}
\centering\begin{tikzcd}[row sep=tiny, column sep=scriptsize]
&&\ZpZ&&\\
						&\Z \ar[dr] \ar[ur] 	& 		&	\R \ar[r] & \C\\
    \N \ar[ur] \ar[dr]  &       		&	\Q \ar[ur]\ar[dr] &	&	  	\\
    					&\Q_+ \ar[ur] 	& 		&	\Qp \ar[r]	& \Cp \\
\end{tikzcd}
\end{block}

\end{frame}


%-=-=-=-=-=-=-=-=-=-=-=-=-=-=-=-=-=-=-=-=-=-=-=-=
%
%	SECTION: BACKGROUND
%
%-=-=-=-=-=-=-=-=-=-=-=-=-=-=-=-=-=-=-=-=-=-=-=-=

\section{$p$-адическая $\zeta$-функция}

%-=-=-=-=-=-=-=-=-=-=-=-=-=-=-=-=-=-=-=-=-=-=-=-=
%	FRAME: What is Beamer?
%-=-=-=-=-=-=-=-=-=-=-=-=-=-=-=-=-=-=-=-=-=-=-=-=

\begin{frame}{Основные понятия}

\begin{exampleblock}{Дзета-функция Римана}
\begin{center}
$\zeta (s)=\sum \limits_{n=1}^{\infty }{\frac {1}{n^{s}}}$
, или же в интегральном виде $\int \limits_1^{\infty}\frac{dx}{x^s} $
\end{center}
\end{exampleblock}

\begin{alertblock}{Утверждение}
\begin{center}
 Пусть $ \mathbb{Q}_p{\stackrel {f}{\longrightarrow }}\mathbb{Q}_p$, такая что $f(2k) = C' \cdot \zeta(2k)$, \\ тогда $f(2k)$ - всегда рационально \\и $\forall \varepsilon > 0$ $ \exists k, k' : |k - k'|_p < \varepsilon$, то $|f(k) - f(k')|_p < \varepsilon$  
\end{center}
\end{alertblock}

\end{frame}{}

\begin{frame}{Числа Бернулли}
\begin{block}{Числа Бернулли}
По определению $\frac{t}{e^t-1} = \frac{1}{1+t/2!+t^2/3! +...+t^n/(n+1)!} = \sum \limits_{k=0}^{\infty}B_k\cdot t^k / k!$ \\ Здесь $B_k$ - числа Бернулли\\ например: $B_0 = 1, B_1 = -1/2, B_2 = 1/6, B_3 = 0, ...$
\\ В $\mathbb{R}$ справедлива следующая формула: $\zeta(2k) = \pi^{2k} \cdot C \cdot B_{2k}$, для $2k>1$
\end{block}
\end{frame}

\begin{frame}{Числа Бернулли}
\begin{exampleblock}{Связь $\zeta(s)$ и $\zeta(s)_p$}
\begin{center}
$\zeta(1-k) = \zeta(1-k)_p/(1-p^{k-1})$, \\ 
$\zeta(1-k) = \prod \limits_{простые q \ne p} \frac{1}{1-q^{k-1}} $

\end{center}

\end{exampleblock}

Практические вычисления \\
\begin{tabular}{|c|c|c|c|c|c|c|c|c|}
\hline
$1-k$ & -1&-3&-5&-7&-9&-11&-13&-15 \\ \hline
$\zeta(1-k)$ & $-\frac{1}{12}$ &$ \frac{1}{120}$ & $-\frac{1}{252}$ & $\frac{1}{240}$ & $-\frac{1}{132}$ & $-\frac{691}{32760}$ & $-\frac{1}{12}$ &$ \frac{3617}{8160}$ \\ \hline
\end{tabular}
\end{frame}


%-=-=-=-=-=-=-=-=-=-=-=-=-=-=-=-=-=-=-=-=-=-=-=-=
%
%	SECTION: UPDATESS
%
%-=-=-=-=-=-=-=-=-=-=-=-=-=-=-=-=-=-=-=-=-=-=-=-=

\section{Практическое применение}

%-=-=-=-=-=-=-=-=-=-=-=-=-=-=-=-=-=-=-=-=-=-=-=-=
%	FRAME:
%-=-=-=-=-=-=-=-=-=-=-=-=-=-=-=-=-=-=-=-=-=-=-=-=

\begin{frame}{Применение в теории суперструн}
Интерполяция дзета-функции находит свое применение при расчете упорядочивающей константы.

$$ \zeta(-1) = \frac{-1}{12} = 1 + 2 + 3 + 4 + 5 + ...$$
\[
\left\{
\begin{aligned}	
& a = \frac{1}{2}(D-2)\sum \limits_{p = 1}^{\infty} p\\
& D = 26, a = -1\\
\end{aligned}
\right.
\]

Значение D приводит к согласованости определения оператора Вирасоро, а значения $a$ к тому, что спектр открытых струн     включал безмассовые фотонные состояния!
\vspace{20mm}
\end{frame}



\begin{frame}{Литература}
	\begin{thebibliography}{10}

	\beamertemplatebookbibitems
	\bibitem{etg} Каток С.Б. p-адический анализ в сравнении с вещественным -- М., МЦНМО, 2004, 112 с. 

	\beamertemplatebookbibitems
	\bibitem{EBU2011}Грин М., Шварц Дж., Виттен Э. Теория суперструн -- М.: Мир, 1990. - 656 с. 

	\beamertemplatebookbibitems
	\bibitem{EBU2012}Цвибах Б. Начальный курс теории струн -- М.: Едиториал УРСС, 2011. - 784 с.	

  \end{thebibliography}
\end{frame}


\end{document}